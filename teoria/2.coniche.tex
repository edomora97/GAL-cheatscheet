\section{Coniche}

\textbf{Ellisse}: luogo dei punti dove la somma delle distanze da due punti fissi (fuochi) è costante. $x^2/a^2 + y^2/b^2 = 1$.

\textbf{Circonferenza}: $x^2+y^2=r^2$

\textbf{Iperbole}: luogo dei punti dove la differenze delle distanze da due punti fissi (fuochi) è costante. $x^2/a^2 - y^2/b^2 = 1$

\textbf{Parabola}: luogo dei punti dove la somma delle distanze da un punto fisso (fuoco) e da una retta fissa (direttrice) è costante. $x^2+2py=0$ dove $p=\pm \sqrt{-\nicefrac{I_3}{\lambda_1}}$

L'eccentricità è il rapporto delle distanze da un fuoco e dalla direttrice. $e = 0$ circonferenza, $0 < e < 1$ ellisse, $e = 1$ parabola, $e > 1$ iperbole. 

\begin{tabular}{lll}
	$I_1$ & $\tr A$ & Invariante lineare \\
	$I_2$ & $\det A$ & Invariante quadratica \\
	$I_3$ & $\det B$ & Invariante cubica
\end{tabular}

\begin{tabular}{c|c|c|l}
	\boldmath$I_3$ & \boldmath$I_2$ & \boldmath$I_1$ & \textbf{Conica} \\
	\hline
	0     & -     &       & Rette reali incidenti \\
	0     & 0     &       & Rette reali parallele o compl. coniugate \\
	0     & +     &       & Rette imm. coniugate incidenti \\
	\hline
	$\emptyset$ & -     & 0     & Iperbole equilatera \\
	$\emptyset$ & -     & $\emptyset$ & Iperbole non equilatera \\
	$\emptyset$ & 0     &       & Parabola \\
	$\pm$ & +     & $\mp$ & Ellisse reale \\
	$\pm$ & +     & $\pm$ & Ellisse immaginaria \\
	\hline
\end{tabular}

\textbf{Centro}: $A\vec{x} = -\vec{b}$

\textbf{Assi}: paralleli agli autovett di $\lambda_1$ e $\lambda_2$

\textbf{Asintoti}: paralleli alle rette in cui si spezzano i termini di secondo grado

\textbf{Semiassi}: $\sqrt{\nicefrac{I_3}{I_2\lambda_i}}$ (con $i=1,2$)

\textbf{Asse della parabola}: direzione dell'autovett di 0

\textbf{Fascio di coniche}: $\lambda F(x,y) + \mu G(x,y) = 0$ (4 punti)
