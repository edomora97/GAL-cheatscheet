\section{Spazi euclidei}

Uno spazio euclideo è uno spazio vettoriale su $\mathbb{R}$ dotato di prodotto scalare.

\begin{tabular}{ll}
	Norma & $||\vec{v}|| = \sqrt{\product{\vec{v}}{\vec{v}}}$ \\
	Versore & $||\vec{u}|| = 1$ \\
	Ortogonalità & $\vec{u} \perp \vec{v} \Leftrightarrow \product{\vec{u}}{\vec{v}} = 0$ \\
	Carnot & $||\vec{u}+\vec{v}||^2 = ||\vec{u}||^2+||\vec{v}||^2+2\product{\vec{u}}{\vec{v}}$ \\
	Pitagora & $\vec{u} \perp \vec{v} \Leftrightarrow ||\vec{u}+\vec{v}||^2 = ||\vec{u}||^2+||\vec{v}||^2$ \\
	Distanza & $||\vec{u}-\vec{v}||$ \\
	Proiezione $\vec{v}$ su $\vec{b}$ & $\vec{v}' = \frac{\product{\vec{v}}{\vec{b}}}{||\vec{b}||^2}\vec{b}$ (coefficiente di Fourier) \\
	Angolo & $\cos \theta = \frac{\product{\vec{u}}{\vec{v}}}{||\vec{u}||\,||\vec{v}||}$ \\
	Dis. triangolare & $||\vec{u}+\vec{v}|| \le ||\vec{u}||+||\vec{v}||$ \\
	Dist. punto-piano & $\frac{|ax_0 + by_0 + cz_0 + d|}{\sqrt{a^2 + b^2 + c^2}}$ \\
	Dist. punto-retta & $\frac{||\vec{u} \times \overline{P_0Q}||}{||\vec{u}||}$ \\
	Dist. retta-retta & $\frac{|\product{\overline{P_1P_2}}{\vec{r_1} \times \vec{r_2}}|}{||\vec{r_1} \times \vec{r_2}||}$ ($P_1 \in r_1$, $P_2 \in r_2$)
\end{tabular}

\textbf{Norma}: $f: V \rightarrow \mathbb{R}$ t.c.
\begin{tabular}{lll}
	$||t\vec{v}|| = |t|\,||\vec{v}||$ &
	$||\vec{v}|| = 0$ sse $\vec{v} = \vec{0}$ &
	$||\vec{u}+\vec{v}|| \le ||\vec{u}||+||\vec{v}||$
\end{tabular}

\textbf{Base ortonormale}: $\Base = \{\vec{q}_1, \vec{q}_2, \dots, \vec{q}_n\}$ dove $\delta_{ij} = \product{\vec{q}_i}{\vec{q}_j}$ con $\delta_{ij} = 0$ quando $i \ne j$ e $\delta_{ij} = 1$ quando $i = j$

Se $U = [\vec{q}_1 | \cdots | \vec{q}_n] \Rightarrow$ U ortogonale.
