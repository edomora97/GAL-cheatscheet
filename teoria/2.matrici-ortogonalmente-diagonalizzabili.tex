\section{Matrici ortogonalmente diagonalizzabili}

A è ort. diag. se esiste U ortogonale t.c. $U^{-1}AU = U^TAU = \diag(\lambda_1, \dots, \lambda_n)$.
A è ort. diag. sse ammette una base ortonorm. di autovett di A.

Se A è simmetrica ha solo autoval reali. Autovett di autoval distinti sono ortogonali (se A è simm.).

\textbf{Teorema spettrale}: A è ort. diag. sse è reale e simmetrica.

Ogni autoval di una matrice simmetrica è regolare.

\textbf{Decomposizione spettrale}: A reale simmetrica, $\lambda_1, \dots, \lambda_h$ autoval distinti, $P_1, \dots, P_h$ matrici proiezione ort.
\begin{tabular}{lll}
	$I_n = P_1 + \cdots + P_h$ & $A = \lambda_1P_1 + \cdots + \lambda_hP_h$ & $P_iP_j = 0$
\end{tabular}

U orto. simm. rappresenta una simmetria orto. rispetto $V_1$.
