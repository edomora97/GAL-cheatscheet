\section{Matrici simili}

A e B sono matrici simili se esiste P t.c. $A = P^{-1}BP$. Condizioni necessarie (e sufficienti se $n \le 3$) per A simile B:

\begin{tabular}{l}
	Stesso polinomio caratteristico \\
	Stessi autoval (e risp. molt.), autovett, det. e traccia \\
	Stesso rango
\end{tabular}

A è \textbf{diagonalizzabile} se è simile ad una matrice diagonale. A è diagonaliz. sse ammette $n$ autovett. l.i. $P^{-1}AP = \diag(\lambda_1, \lambda_2, \dots, \lambda_n)$. $P=[ \vec{c}_1 | \cdots | \vec{c}_n ]$ $\vec{c}_i \in V_{\lambda_i}$ e $\vec{c}_i$ l.i. da $\vec{c}_j$.

Se A ha $n$ autoval. distinti allora è diagonaliz. (non necess.). A è diagonaliz. sse tutti i suoi autoval. sono regoalri
