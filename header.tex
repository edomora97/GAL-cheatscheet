\documentclass[10pt,landscape]{article}
\usepackage[utf8]{inputenc}
\usepackage{multicol}
\usepackage{calc}
\usepackage{ifthen}
\usepackage[landscape]{geometry}
\usepackage{hyperref}
\usepackage{amsmath}
\usepackage{amssymb}
\usepackage{tabularx}
\usepackage{caption}
\usepackage{verbatim}
\usepackage{systeme}
\usepackage{nicefrac}
\usepackage{accents}
\usepackage{enumitem}
\usepackage{marvosym}
\usepackage[printwatermark]{xwatermark}
\usepackage{tikz}
\usetikzlibrary{calc,matrix}
\usepackage[compact]{titlesec}
\usepackage{microtype}


% This sets page margins to .5 inch if using letter paper, and to 1cm
% if using A4 paper. (This probably isn't strictly necessary.)
% If using another size paper, use default 1cm margins.
\ifthenelse{\lengthtest { \paperwidth = 11in}}
{ \geometry{top=.5in,left=.5in,right=.5in,bottom=.5in} }
{\ifthenelse{ \lengthtest{ \paperwidth = 297mm}}
	{\geometry{top=1cm,left=1cm,right=1cm,bottom=1cm} }
	{\geometry{top=1cm,left=1cm,right=1cm,bottom=1cm} }
}

% Turn off header and footer
\pagestyle{empty}

% Reduce size of \section e \subsection
\titleformat{\section}{\normalfont\large\bfseries}{\thesection}{1em}{}
\titleformat{\subsection}{\normalfont\normalsize\bfseries}{\thesubsection}{1em}{}
\titlespacing{\section}{0pt}{0ex}{-0.5ex}
\titlespacing{\subsection}{0pt}{0ex}{-0.5ex}

% Define BibTeX command
\def\BibTeX{{\rm B\kern-.05em{\sc i\kern-.025em b}\kern-.08em
		T\kern-.1667em\lower.7ex\hbox{E}\kern-.125emX}}

% Don't print section numbers
\setcounter{secnumdepth}{0}


\setlength{\parindent}{0pt}
\setlength{\parskip}{0pt plus 0.5ex}

% <v,w> prodotto scalare
\newcommand{\product}[2]{\langle\,#1,#2\,\rangle}
% sottolinea i vettori
\renewcommand{\vec}[1]{\underline{#1}}
% funzioni non defininte
\newcommand{\tr}{\mathrm{tr}}
\newcommand{\diag}{\mathrm{diag}}
\newcommand{\rk}{\mathrm{rk}}
\newcommand{\sgn}{\mathrm{sgn}}
\newcommand{\Col}{\mathrm{Col}}
\newcommand{\Row}{\mathrm{Row}}
\newcommand{\Imm}{\mathrm{Im}}

\setlist[itemize]{noitemsep, nolistsep}

% numeri cerchiati
\newcommand*\circled[1]{\tikz[baseline=(char.base)]{
		\node[shape=circle,draw,inner sep=2pt] (char) {#1};}}

\newcommand{\Lin}{\mathcal{L}}
\newcommand{\Base}{\mathcal{B}}
\newcommand{\Case}{\mathcal{C}}

