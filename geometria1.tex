\documentclass[10pt,landscape]{article}
\usepackage[utf8]{inputenc}
\usepackage{multicol}
\usepackage{calc}
\usepackage{ifthen}
\usepackage[landscape]{geometry}
\usepackage{hyperref}
\usepackage{amsmath}
\usepackage{amssymb}
\usepackage{tabularx}
\usepackage{caption}
\usepackage{verbatim}
\usepackage{systeme}
\usepackage{nicefrac}
\usepackage{accents}
\usepackage[printwatermark]{xwatermark}
\usepackage{tikz}
\usetikzlibrary{calc,matrix}

% This sets page margins to .5 inch if using letter paper, and to 1cm
% if using A4 paper. (This probably isn't strictly necessary.)
% If using another size paper, use default 1cm margins.
\ifthenelse{\lengthtest { \paperwidth = 11in}}
	{ \geometry{top=.5in,left=.5in,right=.5in,bottom=.5in} }
	{\ifthenelse{ \lengthtest{ \paperwidth = 297mm}}
		{\geometry{top=1cm,left=1cm,right=1cm,bottom=1cm} }
		{\geometry{top=1cm,left=1cm,right=1cm,bottom=1cm} }
	}

% Turn off header and footer
\pagestyle{empty}
 

% Redefine section commands to use less space
\makeatletter
\renewcommand{\section}{\@startsection{section}{1}{0mm}%
                                {-1ex plus -.5ex minus -.2ex}%
                                {0.5ex plus .2ex}%x
                                {\normalfont\large\bfseries}}
\renewcommand{\subsection}{\@startsection{subsection}{2}{0mm}%
                                {-1explus -.5ex minus -.2ex}%
                                {0.5ex plus .2ex}%
                                {\normalfont\normalsize\bfseries}}
\renewcommand{\subsubsection}{\@startsection{subsubsection}{3}{0mm}%
                                {-1ex plus -.5ex minus -.2ex}%
                                {1ex plus .2ex}%
                                {\normalfont\small\bfseries}}
\makeatother

% Define BibTeX command
\def\BibTeX{{\rm B\kern-.05em{\sc i\kern-.025em b}\kern-.08em
    T\kern-.1667em\lower.7ex\hbox{E}\kern-.125emX}}

% Don't print section numbers
\setcounter{secnumdepth}{0}


\setlength{\parindent}{0pt}
\setlength{\parskip}{0pt plus 0.5ex}

% <v,w> prodotto scalare
\newcommand{\product}[2]{\langle\,#1,#2\,\rangle}
% sottolinea i vettori
\renewcommand{\vec}[1]{\underline{#1}}
% funzioni non defininte
\newcommand{\rk}{\mathrm{rk}}
\newcommand{\Col}{\mathrm{Col}}
\newcommand{\Row}{\mathrm{Row}}
\newcommand{\Imm}{\mathrm{Im}}

% numeri cerchiati
\newcommand*\circled[1]{\tikz[baseline=(char.base)]{
		\node[shape=circle,draw,inner sep=2pt] (char) {#1};}}

\newcommand{\Lin}{\mathcal{L}}
\newcommand{\Base}{\mathcal{B}}
\newcommand{\Case}{\mathcal{C}}

%\newwatermark[allpages,color=black!10,angle=45,scale=6,xpos=-20,ypos=15]{BOZZA}

% -----------------------------------------------------------------------

\begin{document}

\raggedright
\footnotesize
\begin{multicols}{3}


% multicol parameters
% These lengths are set only within the two main columns
%\setlength{\columnseprule}{0.25pt}
\setlength{\premulticols}{1pt}
\setlength{\postmulticols}{1pt}
\setlength{\multicolsep}{1pt}
\setlength{\columnsep}{2pt}

\begin{center}
     \Large{\textbf{Geometria e Algebra Lineare}} \\
\end{center}

\section{Vettori}
\begin{tabular}{@{}ll@{}}
	$\vec{v} = (v_1, v_2, ..., v_n)$ & Vettore \\
	$||\vec{v}|| = \sqrt{v_1^2 + v_2^2 + ... + v_n^2}$ & Modulo o norma \\
	$
		\begin{bmatrix}
			v_1 \\[-0.3em]
			v_2 \\[-0.3em]
			\cdots \\[-0.3em]
			v_n \\
		\end{bmatrix}
		+
		\begin{bmatrix}
			w_1 \\[-0.3em]
			w_2 \\[-0.3em]
			\cdots \\[-0.3em]
			w_n \\
		\end{bmatrix}
		=
		\begin{bmatrix}
			v_1 + w_1 \\[-0.3em]
			v_2 + w_2 \\[-0.3em]
			\cdots \\[-0.3em]
			v_n + w_n \\
		\end{bmatrix}
	$ & Somma tra vettori \\
	$k \vec{v} = (k v_1, k v_2, ..., k v_n)$ & Prodotto scalare-vettore \\
	$\vec{v} \cdot \vec{w} = v_1 w_1 + v_2 w_2 + ... + v_n w_n$ & Prodotto scalare \\
	$
	\begin{bmatrix}
		x_1 \\[-0.3em]
		y_1 \\[-0.3em]
		z_1 \\
	\end{bmatrix}
	\times
	\begin{bmatrix}
		x_2 \\[-0.3em]
		y_2 \\[-0.3em]
		z_2 \\
	\end{bmatrix}
	=
	\begin{bmatrix}
		y_1z_2 - z_1y_2 \\[-0.3em]
		z_1x_2 - x_1z_2 \\[-0.3em]
		x_1y_2 - y_1x_2 \\
	\end{bmatrix}
	$ & Prodotto vettoriale \\
	$||\vec{v}+\vec{w}||^2 = ||\vec{v}||^2 + ||\vec{w}||^2$ & Se $v \perp w$ \\
	$\cos{\theta} = \frac{\vec{v} \cdot \vec{w}}{||\vec{v}|| \cdot ||\vec{w}||}$ & Angolo tra vettori \\
	$\vec{v} \perp \vec{w} \iff \vec{v} \cdot \vec{w} = 0$ & Vettori perpendicolari \\
	$\vec{v} \parallel \vec{w} \iff \vec{v} \times \vec{w} = \vec{0}$ & Vettori paralleli \\
\end{tabular}
\section{Gruppi}
Un'operazione (+) in uno spazio (V) è un gruppo se:
\begin{tabular}{ll}
	Associativa & $(\vec{v} + \vec{w}) + \vec{z} = \vec{v} + (\vec{w} + \vec{z})$ \\
	Elemento neutro & $\exists \vec{0} \quad \vec{v} + \vec{0} = \vec{v}$ \\
	Inverso & $\exists -\vec{v} \quad \vec{v} + (-\vec{v}) = \vec{0}$
\end{tabular}

Un gruppo è \emph{abeliano} se è un gruppo e:
\begin{tabular}{ll}
	Commutativa & $\vec{v}+\vec{w} = \vec{w}+\vec{v}$ \\
\end{tabular}
\section{Proprietà del prodotto scalare-vettore}
\begin{tabular}{ll}
	$(t_1+t_2)\vec{v} = t_1\vec{v} + t_2\vec{v}$ & $(t_1t_2)\vec{v} = t_1(t_2\vec{v})$ \\
	$t(\vec{v}_1\vec{v}_2) = t\vec{v}_1 + t\vec{v}_2$ & $1\vec{v} = \vec{v}$
\end{tabular}
\section{Proprietà del prodotto scalare}
\begin{tabular}{l}
	$\product{\vec{v}}{\vec{w}} = \product{\vec{w}}{\vec{v}}$ \\
	$\product{t\vec{v}}{t\vec{w}} = t\product{\vec{v}}{\vec{w}}$ \\
	$\product{\vec{v}_1 + \vec{v}_2}{\vec{w}} = \product{\vec{v}_1}{\vec{w}} + \product{\vec{v}_2}{\vec{w}}$ \\
	$\product{t_1\vec{v}_1 + t_2\vec{v}_2}{\vec{w}} = t_1\product{\vec{v}_1}{\vec{w}} + t_2\product{\vec{v}_2}{\vec{w}}$ \\
	$\product{\vec{v}}{\vec{v}} \ge 0$ \\
	$\product{\vec{u}}{\vec{v} \times \vec{w}} = 0$ sse sono l.d. \\
\end{tabular}
\section{Combinazione lineare}
$\vec{w}$ è combinazione lineare di $\vec{v}_1, \vec{v}_2, ..., \vec{v}_n$ sse $\exists a_1, a_2, ..., a_n$
tali che $\vec{w} = a_1\vec{v}_1 + a_2\vec{v}_2 + ... + a_n\vec{v}_n$

$\vec{0}$ è combinazione lineare di un qualunque insieme di vettori.

\section{Dipendenza lineare}
$\vec{v}_1, \vec{v}_2, ..., \vec{v}_n$ si dicono \emph{linearmente dipendenti} sse $\exists a_1, a_2, ..., a_n$
\underline{non tutti nulli} t.c. $a_1\vec{v}_1 + a_2\vec{v}_2 + ... + a_n\vec{v}_n = \vec{0}$

$\vec{v}_1, \vec{v}_2, ..., \vec{v}_n$ si dicono \emph{linearmente indipendenti} se non sono linearmente
dipendenti.

Se un insieme di vettori è linearmente indipendente $a_1 = a_2 = ... = a_n = 0$.

\section{Basi e generatori}
Un insieme di vettori è un generatore sse ogni vettore dello spazio è combinazione lineare del generatore.

Una base è un generatore formato da vettori l.i.

$\dim V = $ numero di vettori di una base qualunque.
$\dim \{\vec{0}\} = 0$.
\section{Spazio vettoriale}
Un insieme di oggetti in cui è definita l'operazione di somma come \emph{gruppo abeliano} e l'operazione
prodotto scalare-vettore che rispetta le proprietà sopraindicate viene definito \emph{spazio vettoriale}.

Esempi di spazi vettoriali: vettori liberi, matrici, polinomi, funzioni lineari.

$\Lin(\vec{v}_1, \vec{v}_2, ..., \vec{v}_n)$ è lo spazio vettoriale generato dalla combinazione lineare
di $\vec{v}_1, \vec{v}_2, ..., \vec{v}_n$.
\section{Sottospazio vettoriale}
$U \subset V$ \quad U è sottospazio di V se è uno spazio vettoriale rispetto le operazioni di V.

\begin{tabular}{llll}
	$\vec{u}_1 + \vec{u}_2 \in U$ &
	$k\vec{u} \in U$ &
	$\vec{0} \in U$ &
	$-\vec{u} \in U$ \\
\end{tabular}

\section{Operazioni sugli spazi vettoriali}
$U$, $W$ sottospazi di $V$ $\Rightarrow$ $U \cap W$ è sottospazio di $V$.

$U+W = \{\vec{v} \in V | \vec{v} = \vec{u} + \vec{w}, \vec{u} \in U, \vec{w} \in W\}$ sottospazio di V.

$|U+W| = |U| + |W| - |U \cap W|$ \underline{Formula di Grassmann}

\vspace{0.6em}

$U \oplus W: \forall \vec{v} \in V\ \exists \vec{u} \in U, \vec{w} \in W\ \text{t.c.}\ \vec{v} = \vec{u} + \vec{w}$ in modo unico

\(
\begin{cases}
U \cap W = \{0\} \\[0em]
U + W = V
\end{cases}
\)
\begin{tabular}{@{}l@{}}
	W è generato dai vettori che aggiunti \\ alla base di U formano una base di V.
\end{tabular}
\section{Trovare una base}
Se $\vec{v}_1, \vec{v}_2, ..., \vec{v}_n$ sono generatori, la base è un sottoinsieme l.i. di quei vettori.
Rimuovo i vettori nulli, da sinistra tolgo i vettori linearmente dipendenti dai precedenti presi.

Se non sono generatori accosto a destra i vettori di una base qualunque e ottengo un insieme di generatori.

\section{Rette}
\begin{tabular}{cc}
	$\begin{cases}
	x = x_0 + at \\[-0.3em]
	y = y_0 + bt \\[-0.3em]
	z = z_0 + ct \\
	\end{cases}$ &
	$\begin{cases}
		a_1x + b_1y + c_1z + d_1 = 0 \\[-0.3em]
		a_2x + b_2y + c_2z + d_2 = 0 \\
	\end{cases}$ \\
	Forma parametrica & Forma cartesiana \\
\end{tabular}

In 3 dimensioni una retta è individuata dall'intersezione di 2 piani non paralleli. Nella forma parametrica
$(a, b, c)$ sono i parametri direttori della retta e ne indicano la direzione.

\section{Piani}
\begin{tabular}{cc}
	$\begin{cases}
		x = x_0 + a_1t + a_2s \\[-0.3em]
		y = y_0 + b_1t + b_2s \\[-0.3em]
		z = z_0 + c_1t + c_2s \\
	\end{cases}$ &
	$ax + by + cz + d = 0$ \\
	Forma parametrica & Forma cartesiana \\
\end{tabular}

Nella forma cartesiana $(a, b, c)$ sono i parametri direttori del piano e indicano la direzione
\emph{perpendicolare} al piano.

\section{Posizioni reciproce piano-retta}
Piano $\pi$ con parametri $\vec{n}$ e retta $r$ con parametri $\vec{d}$

\begin{tabular}{lll}
	$\pi \perp r$ & $\vec{n} \parallel \vec{d}$ & $\vec{n} = k\vec{d}$ \\
	$\pi \parallel r$ & $\vec{n} \perp \vec{d}$ & $\vec{n} \cdot \vec{d} = 0$ \\
	$r_1 \parallel r_2$ & $\vec{d}_1 = k\vec{d}_2$ & \\
	$r_1 \perp r_2$ & $\vec{d}_1 \cdot \vec{d}_2 = 0$ & \\
\end{tabular}
\section{Fasci di piani}
\begin{tabular}{lll}
	Proprio & $\lambda\pi_1 + \mu\pi_2 = 0$ & Piani che contengono $\pi_1 \cap \pi_2$ \\
	Improprio & $\lambda\pi = 0$ & Piani paralleli a $\pi$ \\
\end{tabular}
\section{Matrici}
$A_{m,n} \cdot B_{n,p} = C_{m,p}$

$C_{i,j} = \sum_{k=1}^{n} a_{ik} b_{kj}$ ($i$-esima riga di A $\cdot$ $j$-esima colonna di B)

\begin{tabular}{ll}
	Non commutativa & $A B \ne B A$ \\
	Associativa & $A(BC) = (AB)C$ \\
	Elemento neutro & $AI_n = I_nA = A$ \\
	Distributiva & $A(B+C) = AB + AC$ \\
	Trasposta & $A^T = [\alpha_{ij}] \quad \alpha_{ij} = a_{ji}$ \\
	Inversa & $AA^{-1} = A^{-1}A = I_n\quad (\exists\ \text{sse} \det A \ne 0)$ \\
	& $(AB)^{-1} = B^{-1}A^{-1}$ \\
\end{tabular}

$[A|I_n] \rightarrow [I_n|A^{-1}]$ riducendo a scala

$A^{-1} = (\nicefrac{1}{\det{A}})\ \left[b_{ik}\right] \quad b_{ik} = C_{ki}$ (compl. algebrico trasposto)

$A =
\left[\begin{array}{c|c|c|c} \vec{C}_1 & \vec{C}_2 & \cdots & \vec{C}_n \end{array}\right]
=
\left[\arraycolsep=1.7pt\def\arraystretch{1}\begin{array}{c} \vec{R}_1 \\ \hline \vdots \\ \hline \vec{R}_m \end{array}\right]$

$\Row A = \Lin(\vec{R}_1, \vec{R}_2, ..., \vec{R}_m) \quad \Col A = \Lin(\vec{C}_1, \vec{C}_2, ..., \vec{C}_n)$

$\dim \Row A = \dim \Col A = \rk A$
\section{Riduzione a scala, mosse di Gauss, $\rk$}
Una matrice è a scala se $\vec{R}_i = 0 \Rightarrow \vec{R}_{i+1} = 0$ e le posizioni dei primi elementi non nulli (\emph{pivot})
è strettamente crescente.

$\rk A = $ numero di \emph{pivot} di A ridotta a scala.

Mosse di Gauss:

\begin{tabular}{l}
	Moltiplicare una riga per $k \ne 0$ \\
	Sostituire una riga con la c.l. di due righe \\
	Scambiare due righe \\
\end{tabular}

\section{Determinante}
$\det: M_{n,n}(K) \rightarrow K$

\begin{tabularx}{\textwidth}{lX}
	$n = 1$ & $A = [a] \quad \det{A} = a$ \\
	$n > 1$ &
	Ricorsivamente \newline
	$A_{ik}$ ottenuta da $A$ togliendo riga $i$ e colonna $k$ \newline
	$M_{ik} = \det A_{ik}$ (detto minore complementare) \newline
	$C_{ik} = (-1)^{i+k}M_{ik}$ (detto complemento algebrico) \newline
	$\det A = \sum_{i=1}^{n} a_{1i}C_{1i}$ \\
\end{tabularx}

I th. di Laplace: si può usare una riga o una colonna qualsiasi.

\begin{tabular}{l}
	$\det A = \det A^T$ \\
	Se una riga è di zeri: $\det A = 0$ \\
	Se si scambiano 2 righe: $\det A' = -\det A$ \\
	Se due righe parallele sono proporzionali: $\det A = 0$ \\
	Moltiplicando una riga per $t$: $\det A' = t\det A$ \\
	$\det tA = t^n\det A$ \\
	In una matrice triangolare: $\det A = \prod_{i=1}^{n} a_{ii}$ \\
	T. di Binet: $\det AB = \det A \cdot \det B$
\end{tabular}

Regola di Kronecker: se esiste una sottomatrice quadrata $A'_{n,n}$ con $\det A' \ne 0$ allora $\rk A \ge n$.
Se tutte le matrici ottenute per orlatura da $A'$ hanno $\det = 0$ allora $\rk A = n$.

\begin{tabular}{cc}
	$A = \begin{bmatrix}
	a & b \\
	c & d \\
	\end{bmatrix}$ &
	$B = \begin{bmatrix}
	a & b & c \\
	d & e & f \\
	g & h & i \\
	\end{bmatrix}$ \\
	$\det A = ad - bc$ &
	$\det B = aei + bfg + cdh - ceg - afh - bdi$ \\
\end{tabular}
\section{Sistemi lineari}
\begin{tabular}{@{}l@{}l@{}}
	$\begin{cases}
		a_{11}x_1 + \cdots + a_{1n}x_n = b_1 \\[-0.3em]
		a_{21}x_1 + \cdots + a_{2n}x_n = b_2 \\[-0.3em]
		\vdots \\[-0.3em]
		a_{m1}x_1 + \cdots + a_{mn}x_n = b_m \\
	\end{cases}$ &
	$A|\vec{b} = \left[
		\arraycolsep=1.7pt\def\arraystretch{1.2}
		\begin{array}{ccc|c}
			a_{11} & \cdots & a_{1n} & b_1 \\[-0.3em]
			a_{21} & \cdots & a_{2n} & b_2 \\[-0.3em]
			\vdots & \ddots & \vdots & \vdots \\[-0.3em]
			a_{m1} & \cdots & a_{mn} & b_m \\
		\end{array}
		\right]$
\end{tabular}

Se $b_1 = b_2 = ... = b_m = 0$ il sistema si dice omogeneo ed ammette sempre almeno una soluzione ($\vec{0}$).
Le soluzioni non banali vengono chiamate \emph{autosoluzioni}. Riducendo a scala $A|\vec{b}$ si ottiene un
sistema equivalente.

Se $m = n$ e $\det A \ne 0$ per Cramer $x_i = \det A_i / \det A$ con $A_i$ ottenuta da $A$ sostituendo la $i$-esima
colonna con $\vec{b}$.
\section{Teorema di Rouché-Capelli}
Dato un sistema lineare e le matrici associate $A$ e $A|\vec{b}$:
\begin{tabular}{ll}
	$\rk A \ne \rk A|\vec{b}$ & Sistema impossibile \\
	$\rk A = \rk A|\vec{b}$ & Sistema possibile
\end{tabular}

In un sistema possibile in $n$ incognite con $\rk A = r$:
\begin{tabular}{ll}
	$r = n$ & Sis. det. $1!$ soluzione \\
	$r < n$ & Sis. ind. $\infty^{n-r}$ soluzioni ($n-r$ variabili libere) \\
\end{tabular}

Un sistema lineare omogeneo ha autosoluzioni sse $\rk A < n$.

\section{Posizione reciproca tra rette-piani}
Definiamo $n = $ numero variabili, $r = \rk A$ e $s = \rk A|\vec{b}$

Nel piano:
\begin{tabular}{@{}lllll@{}}
	$n = 2$ & $r = 2$ & $s = 3$ & Imp & Più punti di intersezione \\
	        & $r = 2$ & $s = 2$ & Det & Un punto di intersezione \\
	        & $r = 1$ & $s = 2$ & Imp & Rette parallele \\
	        & $r = 1$ & $s = 1$ & Ind & Rette coincidenti \\
\end{tabular}
Nello spazio:
\begin{tabular}{@{}lllll@{}}
	$n = 2$ & $r = 2$ & $s = 3$ & Imp & Retta + piani $\parallel$ ad essa \\
			& $r = 2$ & $s = 2$ & Det & Una retta \\
			& $r = 1$ & $s = 2$ & Imp & Piani paralleli \\
			& $r = 1$ & $s = 1$ & Ind & Piani coincidenti \\
	$n = 3$ & $r = 3$ & $s = 4$ & Imp & Più punti di intersezione \\
			& $r = 3$ & $s = 3$ & Det & Stella di piani \\
			& $r = 2$ & $s = 3$ & Imp & Retta + piani $\parallel$ ad essa \\
			& $r = 2$ & $s = 2$ & Ind & Una retta \\
			& $r = 1$ & $s = 2$ & Ind & Piani paralleli \\
			& $r = 1$ & $s = 1$ & Ind & Piani coincidenti \\
\end{tabular}

\section{Applicazioni lineari}
$f: V \rightarrow W \quad f(v_1+v_2) = f(v_1) + f(v_2) \quad f(kv) = kf(v)$

\begin{tabular}{l@{\quad}l}
	Definizione & $f(k_1v_1 + k_2v_2) = k_1f(v_1) + k_2f(v_2)$ \\
	Nucleo & $\ker f = \{ \vec{v} \in V | f(\vec{v}) = \vec{0}_{W} \}$ \\
		   & $\ker f$ è un sottospazio di V e $\vec{0} \in \ker f$ \\
	Fibra di $\vec{w}$ & $\{ \vec{v} \in V | f(\vec{v}) = \vec{w} \}$ \\
	Immagine & $\Imm f = f(V) = \Lin(C_1, C_2, ..., C_n)$ \\
			 & $f(\vec{0}_V) = \vec{0}_W$ \\
	Forma lineare & $f: V \rightarrow K$ \\
	Endomorfismo & $f: V \rightarrow V$ \\
	Iniettiva & $\ker f = \{\vec{0}\} \quad \rk A = n$ \\
	Suriettiva & $\Imm f = W \quad \rk A = m$ \\
	Biunivoca & Iniettiva + suriettiva (isomorfismo) $\det A \ne 0$ \\
	Somma di f.l. & $(f+g)(\vec{v}) = f(\vec{v}) + g(\vec{v})$ \\
	Scalare per f.l. & $(hf)(\vec{v}) = hf(\vec{v}) \qquad h \in K$ \\
\end{tabular}

T. nullità più rango: $|V| = |\ker f| + |\Imm f|$

\section{Matrice associata}
$f: V \rightarrow W$

\begin{tabular}{ll}
	Base di V & $B = \{\vec{b}_1, \vec{b}_2, ..., \vec{b}_n\}$ \\
	Base di W & $C = \{\vec{c}_1, \vec{c}_2, ..., \vec{c}_m\}$ \\
\end{tabular}

$A = \begin{bmatrix} f(\vec{b}_1)_{|C} & f(\vec{b}_2)_{|C} & \cdots & f(\vec{b}_n)_{|C} \end{bmatrix} \quad f(\vec{v}) = A\vec{v}$

\vspace{0.5em}

Cambiamento di base da $B$ a $C \quad I_{B\rightarrow C}: V \rightarrow V$

$M_{B\rightarrow C} = \begin{bmatrix} {\vec{b}_1}_{|C} & {\vec{b}_2}_{|C} & \cdots & {\vec{b}_n}_{|C} \end{bmatrix}$

$\vec{v}_{|C} = M_{B\rightarrow C} \cdot \vec{v}_{|B} \quad \vec{v}_{|B} = M_{B\rightarrow C}^{-1} \cdot \vec{v}_{|C}$

\vspace{0.5em}

Combinazione di funzioni: $f: V \rightarrow W\ [A] \quad g: W \rightarrow U\ [B]$

$g \circ f: V \rightarrow U\ [C = B \cdot A]$


\section{Esercizi svolti}

\subsection{Prodotto tra matrici}
\(
\begin{bmatrix}
	1 & 4 \\[-0.2em]
	2 & 3 \\[-0.2em]
	1 & 2 \\
\end{bmatrix}
\begin{bmatrix}
	2 & 1 & 0 & 4 \\[-0.2em]
	3 & 5 & 1 & 3 \\
\end{bmatrix}
=
\begin{bmatrix}
	14 & 21 & 4 & 16 \\[-0.2em]
	13 & 17 & 3 & 17 \\[-0.2em]
	8 & 11 & 2 & 10 \\
\end{bmatrix}
\)
\subsection{Retta $r$ per $A(1, 2, 2)$ e $B(3, 1, 0)$}
$\vec{d} = (1-3, 2-1, 2-0) = (-2, 1, 2)$
\begin{tabular}{@{}ll@{}}
	$\begin{cases}
		x = 1 - 2t \\[-0.3em]
		y = 2 + 1t \\[-0.3em]
		z = 2 + 2t \\
	\end{cases}$
	&
	\begin{tabular}{l}
		$t = y-2$ \\
		$\begin{cases}
		x + 2y - 5 = 0 \\[-0.3em]
		z - 2y +2 = 0 \\
		\end{cases}$ \\
	\end{tabular}
\end{tabular}

\subsection{Piano $\pi \perp r$ passante per $C(1, 1, 2)$}
$\vec{n} = \vec{d}_r = (-2, 1, 2)$

$\pi: -2x + y +2z + d = 0$, imponendo il passaggio per $C$:
$-2 + 1 + 4 + d = 0 \implies d = -3$

\subsection{Verifica di sottospazio}
$S = \{ \vec{v} = (x_1, x_2, x_3) \in \mathbb{R}^3\ |\ x_1 + x_2 + x_3 = 0 \}$

$\vec{v} = (x_1, x_2, x_3) \qquad \vec{w} = (y_1, y_2, y_3)$

\begin{tabular}{l}
	$\vec{0} \in S \quad 0 + 0 + 0 = 0$ \\
	$\vec{v}+\vec{w} = (x_1+y_1, x_2+y_2, x_3+y_3)$ \\
	$\quad (x_1+y_1) + (x_2+y_2) + (x_3+y_3) = $ \\
	$\quad = (x_1 + x_2 + x_3) + (y_1+y_2+y_3) = 0$ \\
	$\lambda\vec{v} = (\lambda x_1, \lambda x_2, \lambda x_3)$ \\
	$\quad \lambda x_1 + \lambda x_2 + \lambda x_3 = \lambda(x_1+x_2+x_3) = 0$
\end{tabular}
\subsection{Risoluzione di sistemi}
\begin{tabular}{ll}
	\(
		\begin{cases}
			2x+4y+4z=4 \\[-0.3em]
			x-z=1 \\[-0.3em]
			-x+3y+4z=2 \\
		\end{cases}
	\)
	&
	\(
	A|\vec{b} = \left[
	\arraycolsep=2.5pt\def\arraystretch{1.2}
	\begin{array}{@{}ccc|c}
		2 & 4 & 4 & 4 \\[-0.3em]
		1 & 0 & -1 & 1 \\[-0.3em]
		-1 & 3 & 4 & 2 \\
	\end{array}
	\right]
	\)
\end{tabular}

Riducendo $A|\vec{b}$ a scala

\(
\left[
	\arraycolsep=2.5pt\def\arraystretch{1.2}
	\begin{array}{@{}ccc|c}
		2 & 4 & 4 & 4 \\[-0.3em]
		1 & 0 & -1 & 1 \\[-0.3em]
		-1 & 3 & 4 & 2 \\
	\end{array}
\right]
\rightarrow
\left[
	\arraycolsep=2.5pt\def\arraystretch{1.2}
	\begin{array}{@{}ccc|c}
		1 & 0 & -1 & 1 \\[-0.3em]
		1 & 2 & 2 & 2 \\[-0.3em]
		-1 & 3 & 4 & 2 \\
	\end{array}
\right]
\rightarrow
\left[
	\arraycolsep=2.5pt\def\arraystretch{1.2}
	\begin{array}{@{}ccc|c}
		1 & 0 & -1 & 1 \\[-0.3em]
		0 & 2 & 3 & 1 \\[-0.3em]
		0 & 3 & 3 & 3 \\
	\end{array}
\right]
\rightarrow
\left[
	\arraycolsep=2.5pt\def\arraystretch{1.2}
	\begin{array}{@{}ccc|c}
		1 & 0 & -1 & 1 \\[-0.3em]
		0 & 2 & 3 & 1 \\[-0.3em]
		0 & 1 & 1 & 1 \\
	\end{array}
\right]
\rightarrow
\left[
	\arraycolsep=2.5pt\def\arraystretch{1.2}
	\begin{array}{@{}ccc|c}
		\circled{1} & 0 & -1 & 1 \\[-0.3em]
		0 & \circled{2} & 3 & 1 \\[-0.3em]
		0 & 0 & \circled{1} & -1 \\
	\end{array}
\right]
\rightarrow
\begin{cases}
	x - z = 1 \\[-0.3em]
	2y + 3z = 1 \\[-0.3em]
	z = -1 \\
\end{cases}
\)
\subsection{Verificare se dei vettori sono l.i.}
$v_1 = (1, -3, 7) \quad v_2 = (2, -1, -1) \quad v_3 = (-4, 2, 2)$

\(
\begin{cases}
	x + 2y -4z = 0 \\[-0.3em]
	-3x -y +2z = 0 \\[-0.3em]
	7x -y +2z = 0 \\
\end{cases}
\rightarrow
\begin{cases}
	x = 0 \\[-0.3em]
	y = 2z \\[-0.3em]
	0 = 0 \\
\end{cases}
\rightarrow
t\begin{bmatrix}
	0 \\[-0.3em]
	2 \\[-0.3em]
	1 \\
\end{bmatrix} \forall t \in \mathbb{R}
\) sono l.d.

$0v_1 + 2tv_2 + tv_3 = 0$

Se l'unica soluzione fosse stata $(0,0,0)$ allora sarebbero l.i.
\subsection{Calcolo del determinante}

\(
\left|
	\arraycolsep=1.5pt\def\arraystretch{1.0}
	\begin{array}{cccc}
		2 & 0 & -2 & 1 \\
		3 & \circled{1} & 3 & 4 \\
		-1 & \circled{2} & 0 & 2 \\
		1 & 0 & 1 & -1 \\
	\end{array}
\right|
=
1 (-)^4
	\left|
		\arraycolsep=2.0pt\def\arraystretch{1.0}
		\begin{array}{ccc}
			2 & -2 & 1 \\
			-1 & 0 & 2 \\
			1 & 1 & -1 \\
		\end{array}
	\right|
+ 2 (-)^5
	\left|
		\arraycolsep=2.0pt\def\arraystretch{1.0}
		\begin{array}{ccc}
			2 & -2 & 1 \\
			3 & 3 & 4 \\
			1 & 1 & -1 \\
		\end{array}
	\right|
=
(0-1-4-0+2-4) -2(-6-8+3-3-8-6) = 49
\)

\subsection{Trovare l'inversa di una matrice}
\(
A|I_2 = \left[
	\arraycolsep=2.0pt\def\arraystretch{1.0}
	\begin{array}{cc|cc}
		1 & 2 & 1 & 0 \\
		2 & -1 & 0 & 1 \\
	\end{array}
\right]
\rightarrow
\left[
	\arraycolsep=2.0pt\def\arraystretch{1.0}
	\begin{array}{cc|cc}
		1 & 2 & 1 & 0 \\
		0 & 5 & 2 & -1 \\
	\end{array}
\right]
\rightarrow
\left[
	\arraycolsep=2.0pt\def\arraystretch{1.0}
	\begin{array}{cc|cc}
		1 & 2 & 1 & 0 \\
		0 & 1 & 2/5 & -1/5 \\
	\end{array}
\right]
\rightarrow
\left[
	\arraycolsep=2.0pt\def\arraystretch{1.0}
	\begin{array}{cc|cc}
		1 & 0 & 1/5 & 2/5 \\
		0 & 1 & 2/5 & -1/5 \\
	\end{array}
\right]
\rightarrow
A^{-1} =
\frac{1}{5}
\left[
	\arraycolsep=2.0pt\def\arraystretch{1.0}
	\begin{array}{cc}
		1 & 2 \\
		2 & -1 \\
	\end{array}
\right]
\)

\subsection{Spazi vettoriali}
$V = \Lin(\begin{bmatrix}1 & 1\end{bmatrix}) \quad W = \Lin(\begin{bmatrix}1 & 0\end{bmatrix})$

$V+W = \Lin([1,1], [1,0])$ sono l.i. quindi $|V+W| = 2$

Per Grassmann $|V \cap W| = 0$

\subsection{Applicazioni lineari}
$T: \mathbb{R}^2 \rightarrow \mathbb{R}^2 \quad T(1,1) = (1,2) \quad T(0,2) = (4,4)$

$(x,y) = a(1,1) + b(0,2) \Rightarrow a = x \quad b = \frac{y-x}{2}$

$T(x,y) = aT(1,1) + bT(0,2) = x(1,2) + \frac{y-x}{2}(4,4) = (2y-x, 2y)$

$T(1,0) = (-1,0) \qquad T(0,1) = (2,2)$

$M = \left[
	\arraycolsep=2.0pt\def\arraystretch{1.0}
	\begin{array}{@{}cc@{}}
		-1 & 2 \\
		0 & 2 \\
	\end{array}
\right]
\quad
\ker f = \begin{cases}
	-x +2y = 0 \\
	2y = 0 \\
\end{cases}
= \{\vec{0}\}
$ (iniettiva)

$|\mathbb{R}^2| = |\ker f| + |\Imm f| \Rightarrow 2 = 0 + |\Imm f|$ (nullità$+\rk$)

$\Imm f = \mathbb{R}^2$ (suriettiva, isomorfismo)

\subsection{Applicazione lineare con basi non canoniche}
$T: \mathbb{R}^2 \rightarrow \mathbb{R}^3 \quad T(x,y) = (2x, x-y, 2y)$

$\Base=\{(1,0), (1,1)\} \quad \Base'=\{(1,1,0),(0,1,1),(0,0,2)\}$

$T(1,0) = (2, 1, 0)_{|\Case} = (2, -1, 1/2)_{|\Base'}$

$T(1,1) = (2, 0, 2)_{|\Case} = (2, -2, 2)_{|\Base'}$

$
M = \left[
	\arraycolsep=2.0pt\def\arraystretch{1.0}
	\begin{array}{@{}cc@{}}
		2 & 2 \\
		-1 & -2 \\
		1/2 & 2 \\
	\end{array}
\right]
$
\begin{tabular}{l}
	\emph{Esempio} \\
	$T_{|\Case}(0,1) = T_{|\Base}(-1,1) = M \cdot (-1,1)_{|\Base} =$ \\
	$(0,-1,3/2)_{|\Base'} = (0,-1,2)_{|\Case}$ \\
\end{tabular}

$\ker f = \begin{cases}
2x+2y = 0 \\[-0.3em]
-x-2y = 0 \\[-0.3em]
x/2+2y = 0 \\
\end{cases} = \{\vec{0}\}$ (iniettiva)

$|\Imm f| = 2$ (th. nullità più rango)

$\Imm f = \Col M = t (2, -1, 1/2) + s (2, -2, 2)$

\subsection{Cambiamento di base}
\begin{tabular}{ll}
	$\Base = \{(1,0), (1,1)\}$ & $\Base' = \{(2,1),(1,2)\}$ \\

	$(1,0) = (2/3,-1/3)_{|\Base'}$ & $(1,1) = (1/3,1/3)_{|\Base'}$ \\
\end{tabular}

$M_{\Base\rightarrow \Base'} =
\begin{bmatrix}
	2/3 & 1/3 \\
	-1/3 & 1/3 \\
\end{bmatrix}$

$\left((2,2)_{|\Base}\right)_{|\Base'} = M_{\Base\rightarrow \Base'} \cdot (2,2)_{|\Base} = (2,0)_{|\Base'}$

$M_{\Base'\rightarrow \Base} = M_{\Base\rightarrow \Base'}^{-1} =
\begin{bmatrix}
	1 & -1 \\
	1 & 2 \\
\end{bmatrix}$

$\left((2,0)_{|\Base'}\right)_{|\Base} = M_{\Base'\rightarrow \Base} \cdot (2,0)_{|\Base} = (2,2)_{|\Base'}$


\rule{0.3\linewidth}{0.25pt}
\scriptsize

Copyright \copyright\ 2016 Edoardo Morassutto

\end{multicols}
\end{document}
