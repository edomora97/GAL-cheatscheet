\subsection{Applicazioni lineari}
$T: \mathbb{R}^2 \rightarrow \mathbb{R}^2 \quad T(1,1) = (1,2) \quad T(0,2) = (4,4)$

$(x,y) = a(1,1) + b(0,2) \implies a = x \quad b = \frac{y-x}{2}$

$T(x,y) = aT(1,1) + bT(0,2) = x(1,2) + \frac{y-x}{2}(4,4) = (2y-x, 2y)$

$T(1,0) = (-1,0) \qquad T(0,1) = (2,2)$

$M = \left[
	\arraycolsep=2.0pt\def\arraystretch{1.0}
	\begin{array}{@{}cc@{}}
		-1 & 2 \\
		0 & 2 \\
	\end{array}
\right]
\quad
\ker f = \begin{cases}
	-x +2y = 0 \\
	2y = 0 \\
\end{cases}
= \{\vec{0}\}
$ (iniettiva)

$|\mathbb{R}^2| = |\ker f| + |\Imm f| \Rightarrow 2 = 0 + |\Imm f|$ (nullità$+\rk$)

$\Imm f = \mathbb{R}^2$ (suriettiva, isomorfismo)
