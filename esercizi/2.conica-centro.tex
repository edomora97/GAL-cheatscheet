\subsection{Conica a centro}
$\mathfrak{C}:\ 3x^2-2xy+3y^2-x-y-2=0$
$
A = \left[
	\arraycolsep=2.0pt\def\arraystretch{1.0}
	\begin{array}{@{}cc@{}}
		3  & -1 \\
		-1 & 3
	\end{array}
\right]
\
B = \left[
	\arraycolsep=2.0pt\def\arraystretch{1.0}
	\begin{array}{@{}ccc@{}}
		3  & -1 & -1/2 \\
		-1 & 3  & -1/2 \\
		-1/2  & -1/2 & -2 \\
	\end{array}
\right]
$
\begin{tabular}{l}
	$I_1 = \tr A = 6$ \\
	$I_2 = \det A = 8$ \\ 
	$I_3 = \det B = -18$
\end{tabular}

$I_2 \neq 0 \Rightarrow$ conica a centro. $I_1 I_3 < 0 \Rightarrow$ ellisse reale

Calcolo il centro che soddisfa $A \vec{v} = -\vec{b}$
$
\left[
	\arraycolsep=2.0pt\def\arraystretch{1.0}
	\begin{array}{@{}cc|c@{}}
		3  & -1 & 1/2 \\
		-1 & 3  & 1/2
	\end{array}
\right]
\rightarrow
\left[
	\arraycolsep=2.0pt\def\arraystretch{1.0}
	\begin{array}{@{}cc|c@{}}
		3  & -1   & 1/2 \\
		0  & 8/3  & 2/3 
	\end{array}
\right]
\Rightarrow
O'_{|\Base} =
\begin{bmatrix}
	1/4 \\[-0.3em]
	1/4 \\
\end{bmatrix}
$

Calcolo autospazi di $A$ (autovettori sono paralleli assi di $\mathfrak{C}$)
$p(\lambda) = \det(A-\lambda I) = (2-\lambda)(4-\lambda)$
$\ \Rightarrow \lambda_1 = 2, \lambda_2 = 4$
$V_2 = \Lin 
\begin{bmatrix}
	1 \\[-0.3em]
	1 \\
\end{bmatrix}
= \Lin
\begin{bmatrix}
	1/\sqrt{2} \\[-0.3em]
	1/\sqrt{2} \\
\end{bmatrix}
\quad
V_4 = \Lin 
\begin{bmatrix}
	-1 \\[-0.3em]
	1 \\
\end{bmatrix}
= \Lin
\begin{bmatrix}
	-1/\sqrt{2} \\[-0.3em]
	1/\sqrt{2} \\
\end{bmatrix}
$

Cambio di coordinate (nota che $Q \in SO(n)$):
$
\begin{bmatrix}
	x \\[-0.3em]
	y \\
\end{bmatrix}
= Q
\begin{bmatrix}
	x' \\[-0.3em]
	y' \\
\end{bmatrix}
+ O'_{|\Base} = 
\left[
	\arraycolsep=2.0pt\def\arraystretch{1.0}
	\begin{array}{@{}cc@{}}
		1/\sqrt{2} & -1/\sqrt{2} \\
		1/\sqrt{2} & 1/\sqrt{2}
	\end{array}
\right]
\begin{bmatrix}
	x' \\[-0.3em]
	y' \\
\end{bmatrix}
+
\begin{bmatrix}
	1/4 \\[-0.3em]
	1/4 \\
\end{bmatrix}
$

Essendo un ellisse: $\lambda_1x'^2 + \lambda_2y'^2 + t = 2x'^2 + 4y'^2 + t = 0$
$
B' = \left[
	\arraycolsep=2.0pt\def\arraystretch{1.0}
	\begin{array}{@{}ccc@{}}
		2 & 0 & 0 \\
		0 & 4 & 0 \\
		0 & 0 & t \\
	\end{array}
\right]
$
\begin{tabular}{l}
	$\Rightarrow I_3 = \det B' = 8t = \det B = -18$ \\
	$\Rightarrow t = -9/4$
\end{tabular}
$\Rightarrow \mathfrak{C'}:\ 2x'^2 + 4y'^2 -9/4 = 0$
$\Rightarrow \mathfrak{C'}:\ \frac{8}{9}x'^2 + \frac{16}{9}y'^2 = 1$
